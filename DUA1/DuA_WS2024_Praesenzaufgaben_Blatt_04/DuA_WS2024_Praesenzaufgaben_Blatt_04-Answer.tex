\documentclass[11pt]{article}

\usepackage{amsmath}
\usepackage{amssymb}
\usepackage{algorithm}
\usepackage{algpseudocode}
\usepackage{enumitem}
\usepackage{fullpage}

\topmargin -.5in
\textheight 9in
\oddsidemargin -.25in
\evensidemargin -.25in
\textwidth 7in

\begin{document}

\author{Konstantin Krasser}
\title{DUA: Assignment 1}
\maketitle

\medskip

\section*{Aufgabe 1 (9 points)}

In the lecture, you learned about the partition function. This function can not only be used to sort an array $A$ (as in quicksort), but also to find the $k$-th smallest value in $A$, i.e., the $k$-th element in the ascending sorted order of the elements in $A$.

Example: In the array $A=[6,7,2,9,3,1,0]$, the 4th smallest element is the number 3 (only 0, 1, and 2 are smaller). We assume that every element in $A$ is unique.

\begin{enumerate}

	\item Answer to question 1:

	      What would a naive approach, using comparison-based search algorithms, look like to find the $k$-th smallest element in $A$? What is the lower asymptotic bound on the runtime of this approach?

	      The naive approach would be to sort the array and then return the $k$-th element. The time complexity for comparison-based sorting algorithms is:
	      \begin{equation}
		      O(n \log n)
	      \end{equation}

	\item Answer to question 2:

	      Provide a modified partition function in pseudocode that rearranges the array $A$ such that the pivot element is at position $i_p$, and all elements to the left of $i_p$ are smaller than the pivot element, and all elements to the right of $i_p$ are greater than the pivot element. The manipulation of $A$ is done in-place, so all changes are made directly in $A$, and partition does not need to explicitly return the array $A$, only $i_p$.
	      \begin{enumerate}[label=\textbf{[\arabic*]}]
		      \item $p \leftarrow \mathcal{A}[n-1]$
		      \item $i \leftarrow -1$
		      \item \textbf{for} $j \leftarrow 0$ \textbf{to} $n - 2$ \textbf{do}
		            \begin{enumerate}[label=\textbf{[\arabic*]}, leftmargin=2\parindent]
			            \item \textbf{if} $\mathcal{A}[j] \leq p$ \textbf{then}
			                  \begin{enumerate}[label=\textbf{[\arabic*]}, leftmargin=3\parindent]
				                  \item $\operatorname{swap}(\mathcal{A}[i+1], \mathcal{A}[j])$
				                  \item $i \leftarrow i+1$
			                  \end{enumerate}
		            \end{enumerate}
		      \item $\operatorname{swap}(\mathcal{A}[i+1], \mathcal{A}[n-1])$
		      \item $i \leftarrow i+1$
		      \item \textbf{return} $(\mathcal{A}, i)$
	      \end{enumerate}

	\item Answer to question 3:

	      Describe in words and in pseudocode how the modified partition function can be used to efficiently find the $k$-th smallest value in $A$.

	      Was passiert hier? Das Pivotelement $p$ ist das letzte Element von $\mathcal{A}$, siehe Zeile [1]. Wie oben erwähnt, partionieren wir das Array in Elemente, welche kleiner oder gleich $p$ sind, sowie die Elemente, welche größer als $p$ sind. Das Pivotelement $p$ dient erstmal nur zum Vergleichen ([4]), bleibt aber ansonsten außen vor, bis es in Zeile [7] an seine endgültige Position getauscht wird.

	      Die endgültige Position von $p$ wird von dem Index $i$ bestimmt. Dabei erfüllt $i$ zu jedem Zeitpunkt die Bedingung, dass alles, was sich links von $i$ befindet ( $i$ eingeschlossen), stets kleiner oder gleich $p$ ist (Zeile [4] und [5])! Unter den Elementen, die bereits mit $p$ verglichen wurden (siehe Laufindex $j$ ) nimmt $i$ dabei den maximalen Wert ein (Nach Ausführung von Zeile [6], bzw Zeile [8]). Zu beachten ist außerdem, dass durch das rechtzeitige Addieren von $i+1$ nie ein Arrayzugriff an undefinierter Stelle geschieht([5]), selbst wenn $i$ mit -1 initialisiert wurde([2]).

	      Als letztes muss nur noch die Rolle des Laufindex $j$ geklärt werden. Definiert in Zeile [3] startet $j$ beim ersten Element und geht bis zum vorletzten (also exklusiv $p$ ). Da pro Schleifendurchgang $i$ um maximal eins inkrementiert werden kann, ist $j$ also stets größergleich $i$. Dabei zeigt $j$ an, welche Elemente des Arrays bereits mit $p$ verglichen wurden. Findet $j$ mit Zeile [4] ein Element, welches kleiner ist als $p$, wird dieses in den von $i$ markierten Bereich vertauscht([5]).



	\item Answer to question 4:

	      What is the runtime of your algorithm in the best case and in the worst case? Provide an example call for both cases. The best/worst case should apply to general $k$, not a specific $k$.

	      The \textbf{best-case} runtime occurs when the pivot always splits the array evenly, giving a time complexity of $O(n)$, since each partition step reduces the problem size by half.

	      The \textbf{worst-case} runtime occurs when the pivot is always the smallest or largest element, leading to a time complexity of $O(n^2)$ because each partition only reduces the problem size by one element.

	      Example best-case: \texttt{quickselect([1,2,3,4,5], 0, 4, 2)}.

	      Example worst-case: \texttt{quickselect([5,4,3,2,1], 0, 4, 2)}.

\end{enumerate}

\end{document}
