\documentclass[11pt]{article}

\newcommand{\numpy}{{\tt numpy}}    % tt font for numpy

\topmargin -.5in
\textheight 9in
\oddsidemargin -.25in
\evensidemargin -.25in
\textwidth 7in

\begin{document}

% ========== Edit your name here
\author{Konstantin Krasser}
\title{DUA: }
\maketitle

\medskip

% ========== Begin answering questions here
Aufgabe 1 (9 points). In the lecture, you learned about the partition function. This function can not only be used to sort an array \(A\) (as in quicksort), but also to find the \(k\)-th smallest value in \(A\), i.e., the \(k\)-th element in the ascending sorted order of the elements in \(A\).
Example: In the array \(A=[6,7,2,9,3,1,0]\), the 4th smallest element is the number 3 (only 0, 1, and 2 are smaller). We assume that every element in \(A\) is unique.
\begin{enumerate}

	\item Answer to question 1:

	      What would a naive approach, using comparison-based search algorithms, look like to find the \(k\)-th smallest element in \(A\)? What is the lower asymptotic bound on the runtime of this approach?

	      The naive approach would be to sort the array and then return the \(k\)-th element. The time complexity for comparison-based sorting algorithms is:
	      \begin{equation}
		      O(n \log n)
	      \end{equation}

	\item Answer to question 2:

	      (2 points) Provide a modified partition function in pseudocode that rearranges the array \(A\) such that the pivot element is at position \(i_p\), and all elements to the left of \(i_p\) are smaller than the pivot element, and all elements to the right of \(i_p\) are greater than the pivot element. The manipulation of \(A\) is done in-place, so all changes are made directly in \(A\), and partition does not need to explicitly return the array \(A\), only \(i_p\).

	      \begin{verbatim}
	      function partition(A, low, high)
	          pivot = A[high]
	          i_p = low - 1
	          for j = low to high - 1 do
	              if A[j] <= pivot then
	                  i_p = i_p + 1
	                  swap(A[i_p], A[j])
	          swap(A[i_p + 1], A[high])
	          return i_p + 1
	      \end{verbatim}

	\item Answer to question 3:

	      (4 points) Describe in words and in pseudocode how the modified partition function can be used to efficiently find the \(k\)-th smallest value in \(A\).

	      The modified partition function can be used to find the \(k\)-th smallest element in an array \(A\) by using a quickselect algorithm. We choose a pivot and partition the array around it. If the pivot index \(i_p\) equals \(k\), we are done. If \(i_p > k\), recursively search in the left subarray, and if \(i_p < k\), recursively search in the right subarray.

	      \begin{verbatim}
	      function quickselect(A, low, high, k)
	          if low == high:
	              return A[low]
	          i_p = partition(A, low, high)
	          if i_p == k:
	              return A[i_p]
	          else if i_p > k:
	              return quickselect(A, low, i_p - 1, k)
	          else:
	              return quickselect(A, i_p + 1, high, k)
	      \end{verbatim}

	\item Answer to question 4:

	      (2 points) What is the runtime of your algorithm in the best case and in the worst case? Provide an example call for both cases. The best/worst case should apply to general \(k\), not a specific \(k\).

	      The best-case runtime occurs when the pivot always splits the array evenly, giving a time complexity of \(O(n)\), since each partition step reduces the problem size by half. The worst-case runtime occurs when the pivot is always the smallest or largest element, leading to a time complexity of \(O(n^2)\) because each partition only reduces the problem size by one element.

	      Example best-case: \(quickselect([1,2,3,4,5], 0, 4, 2)\).

	      Example worst-case: \(quickselect([5,4,3,2,1], 0, 4, 2)\).

\end{enumerate}
\end{document}
