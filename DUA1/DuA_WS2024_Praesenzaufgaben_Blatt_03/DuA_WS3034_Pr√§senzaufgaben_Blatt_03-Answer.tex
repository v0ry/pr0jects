\documentclass[11pt]{article}

\newcommand{\numpy}{{\tt numpy}}    % tt font for numpy

\topmargin -.5in
\textheight 9in
\oddsidemargin -.25in
\evensidemargin -.25in
\textwidth 7in

\usepackage{amsmath}

\begin{document}

% ========== Edit your name here
\author{Konstantin Krasser}
\title{Datenstrukturen und Algorithmen, WS2024, Übungsblatt 2\\Rekursive Laufzeitfunktionen}
\maketitle

\medskip

% ========== Begin answering questions here
\begin{enumerate}

	\item
	      \textbf{Aufgabe 1 (4 Punkte).} Sei \(T(n)=1\) für \(n \leq 2\). Für \(n>2\) gilt: \(T(n)=T(n-2)+\log n\). Zeigen Sie: \(T \in \mathcal{O}(n \log n)\). Sie können annehmen, dass \(n\) gerade ist.

	      \textbf{Lösung:}

	      Um zu zeigen, dass \(T(n) \in \mathcal{O}(n \log n)\), benutzen wir eine Induktion und eine Abschätzung der Summe:

	      \begin{enumerate}
		      \item \textbf{Basisfall:} Für \(n \leq 2\) ist \(T(n) = 1\), was offensichtlich eine konstante Laufzeit ist und somit kleiner als \(\mathcal{O}(n \log n)\) für kleine \(n\).
		      \item \textbf{Induktionsschritt:} Für \(n > 2\) gilt:
		            \[
			            T(n) = T(n-2) + \log n
		            \]

		            Um \(T(n)\) abzuschätzen, summieren wir die Terme bis zum Basisfall:
		            \[
			            \log(2k)=log(2)+log(k)
		            \]
		            \[
			            T(n) = 1 + \sum_{k=2}^{n/2} \log(2k)
		            \]
		            Diese Summe lässt sich durch \(n \log n\) abschätzen. Daher folgt:
		            \[
			            T(n) \in \mathcal{O}(n \log n)
		            \]
	      \end{enumerate}

	\item
	      \textbf{Aufgabe 2 (4 Punkte).} Berechnen Sie eine scharfe asymptotische obere Schranke für \(T(n)=30 T(n / 3)+n^3\). Es gilt für den Basisfall \(n<2: T(n)=1\).

	      \textbf{Lösung:}

	      Wir wenden das Master-Theorem an, um eine scharfe obere Schranke zu finden:

	      \[
		      T(n) = a \cdot T\left(\frac{n}{b}\right) + f(n)
	      \]
	      mit \(a = 30\), \(b = 3\), und \(f(n) = n^3\).

	      \begin{enumerate}
		      \item Berechnen Sie den Exponenten \(\log_b a\):
		            \[
			            \log_3 30 \approx 3.096
		            \]
		      \item Vergleichen Sie \(f(n) = n^3\) mit \(n^{\log_3 30}\):

		            Da \(f(n)\) (\(n^3\)) kleiner ist als \(n^{\log_3 30}\), befinden wir uns im Fall 1 des Master-Theorems:
		            \[
			            T(n) = \Theta\left(n^{\log_3 30}\right)
		            \]
		            Also ist die scharfe asymptotische obere Schranke:
		            \[
			            T(n) = \mathcal{O}\left(n^{3.096}\right)
		            \]
	      \end{enumerate}

	\item
	      \textbf{Aufgabe 3 (4 Punkte).} Berechnen Sie eine möglichst knappe obere Schranke der folgenden rekursiven Funktion \(T(n)\) : \(T(n)=2 T(\sqrt{n})+\log_2 n\), mit \(T(n)=\mathcal{O}(1)\) für \(n \leq 2\).

	      \textbf{Lösung:}

	      Wir wenden eine Veränderliche-Substitution an: Setzen Sie \(m = \log_2 n\), dann ist \(n = 2^m\). Die Rekursion wird umgeschrieben als:

	      \[
		      T(2^m) = 2 T(2^{m/2}) + m
	      \]

	      Setzen Sie \(S(m) = T(2^m)\). Dann haben wir:

	      \[
		      S(m) = 2 S(m/2) + m
	      \]

	      Diese Rekursion entspricht der Form des Master-Theorems mit \(a = 2\), \(b = 2\), und \(f(m) = m\). Da \(f(m) = m = \Theta(m^1)\), haben wir \(d = 1\). Der kritische Exponent ist:

	      \[
		      \log_b a = \log_2 2 = 1
	      \]

	      Da \(f(m)\) der gleichen Ordnung ist (\(m^1\)), befinden wir uns im Fall 2 des Master-Theorems:

	      \[
		      S(m) = \Theta(m \log m)
	      \]

	      Daher ist \(T(n) = S(\log_2 n) = \Theta(\log n \cdot \log \log n)\). Eine möglichst knappe obere Schranke ist also:

	      \[
		      T(n) = \mathcal{O}(\log n \cdot \log \log n)
	      \]

\end{enumerate}

\end{document}

