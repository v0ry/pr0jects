\documentclass[11pt]{article}

\usepackage{amsmath}
\usepackage{amssymb}
\usepackage{geometry}
\geometry{
  a4paper,
  left=25mm,
  right=25mm,
  top=20mm,
  bottom=20mm,
}

\begin{document}

% ========== Edit your name here
\author{Konstantin Krasser}
\title{Datenstrukturen und Algorithmen, WS2024, Übungsblatt 2}
\maketitle

\medskip

Abzugeben bis siehe TC.

\section*{Rekursive Laufzeitfunktionen}

\section*{Hausaufgaben}

% ========== Begin answering questions here
\begin{enumerate}

	\item \textbf{Aufgabe 1 (4 Punkte).} Berechnen Sie eine scharfe asymptotische obere Schranke für \(T(n) = 7 T(n / 2) + n^{2}\). Die Basis ist \(n < 2\): \(T(n) = 1\).

	      \textbf{Lösung:}

	      Wir verwenden das Master-Theorem, um die Rekurrenz zu lösen.

	      Die Rekurrenz hat die Form:
	      \[ T(n) = a\, T\left( \frac{n}{b} \right) + f(n) \]
	      mit
	      \( a = 7 \), \( b = 2 \), und \( f(n) = n^{2} \).

	      Wir berechnen \(\log_b a\):
	      \[ \log_b a = \log_2 7 \approx 2.807 \]

	      Wir vergleichen \(f(n)\) mit \(n^{\log_b a}\):
	      \[ f(n) = n^{2} \]
	      \[ n^{\log_b a} = n^{\log_2 7} \approx n^{2.807} \]

	      Da \( f(n) = O\left( n^{\log_b a - \epsilon} \right) \) für ein \( \epsilon > 0 \) (hier \( \epsilon \approx 0.807 \)), befinden wir uns in Fall 1 des Master-Theorems.

	      Daher gilt:
	      \[ T(n) = \Theta\left( n^{\log_b a} \right) = \Theta\left( n^{\log_2 7} \right) \]

	      Also ist die scharfe asymptotische obere Schranke:
	      \[ T(n) = O\left( n^{\log_2 7} \right) \]

	\item \textbf{Aufgabe 2 (4 Punkte).} Gegeben sei \(T(n) = 1\) für \(n \leq 2\). Für \(n > 2\) gilt: \(T(n) = \sqrt{n}\, T(\sqrt{n}) + n\). Zeigen Sie, dass \(T(n) \in \mathcal{O}(n \log \log n)\).

	      \textit{Hinweis: Betrachten Sie zunächst \(T'(n) = \frac{T(n)}{n}\), woraus folgt, dass \(T'(\sqrt{n}) = \frac{T(\sqrt{n})}{\sqrt{n}}\).}

	      \textbf{Lösung:}

	      Wir definieren \( T'(n) = \frac{T(n)}{n} \). Dann gilt \( T(n) = n\, T'(n) \).

	      Die gegebene Rekurrenz wird zu:
	      \[
		      T(n) = \sqrt{n}\, T(\sqrt{n}) + n
	      \]
	      Dividieren wir beide Seiten durch \( n \):
	      \[
		      \frac{T(n)}{n} = \frac{\sqrt{n}\, T(\sqrt{n})}{n} + 1
	      \]
	      \[
		      T'(n) = \frac{T(\sqrt{n})}{\sqrt{n}} + 1 = T'(\sqrt{n}) + 1
	      \]

	      Wir haben also die Rekurrenz:
	      \[
		      T'(n) = T'(\sqrt{n}) + 1
	      \]

	      Um diese zu lösen, betrachten wir die Iteration:
	      \[
		      \begin{aligned}
			      T'(n) & = T'(\sqrt{n}) + 1            \\
			            & = T'(\sqrt[4]{n}) + 1 + 1     \\
			            & = T'(\sqrt[8]{n}) + 1 + 1 + 1 \\
			            & \vdots                        \\
			            & = T'(2) + k
		      \end{aligned}
	      \]
	      wobei \( k \) die Anzahl der Iterationsschritte ist.

	      Wir stoppen die Iteration, wenn \( \sqrt[2^{k}]{n} \leq 2 \), also wenn:
	      \[
		      \log_2 n^{1/2^{k}} \leq 1 \implies \frac{\log_2 n}{2^{k}} \leq 1
	      \]
	      Daraus folgt:
	      \[
		      2^{k} \geq \log_2 n \implies k \geq \log_2 \log_2 n
	      \]

	      Damit ist:
	      \[
		      T'(n) = T'(2) + k = T'(2) + \log_2 \log_2 n
	      \]
	      Da \( T'(2) \) eine Konstante ist, gilt:
	      \[
		      T'(n) = \mathcal{O}(\log \log n)
	      \]
	      Da \( T(n) = n\, T'(n) \), folgt:
	      \[
		      T(n) = n\, \mathcal{O}(\log \log n) = \mathcal{O}(n \log \log n)
	      \]

	\item \textbf{Aufgabe 3 (4 Punkte).} Berechnen Sie eine scharfe asymptotische obere Schranke für die folgende Rekurrenzfunktion \(T(n) = T(\sqrt{n}) + \log_{2} \log_{2} n\). Nehmen Sie an, dass \(T(n) = 1\) für \(n \leq 2\).

	      \textit{Hinweis: Verwenden Sie die Substitution \(m = \log_{2} n\), ähnlich wie in der Übung.}

	      \textbf{Lösung:}

	      Wir setzen \( m = \log_2 n \). Dann ist \( n = 2^{m} \) und \( \sqrt{n} = 2^{m/2} \). Außerdem gilt \( \log_2 \log_2 n = \log_2 m \).

	      Die Rekurrenz wird zu:
	      \[
		      T(2^{m}) = T\left( 2^{\frac{m}{2}} \right) + \log_2 m
	      \]
	      Definieren wir \( S(m) = T(2^{m}) \), erhalten wir:
	      \[
		      S(m) = S\left( \frac{m}{2} \right) + \log_2 m
	      \]

	      Wir lösen diese Rekurrenz durch Iteration:
	      \[
		      \begin{aligned}
			      S(m) & = S\left( \frac{m}{2} \right) + \log_2 m                                                                         \\
			           & = S\left( \frac{m}{4} \right) + \log_2 \left( \frac{m}{2} \right) + \log_2 m                                     \\
			           & = S\left( \frac{m}{8} \right) + \log_2 \left( \frac{m}{4} \right) + \log_2 \left( \frac{m}{2} \right) + \log_2 m \\
			           & \vdots                                                                                                           \\
			           & = S\left( \frac{m}{2^{k}} \right) + \sum_{i=0}^{k-1} \log_2 \left( \frac{m}{2^{i}} \right)
		      \end{aligned}
	      \]

	      Wir stoppen die Iteration, wenn \( \frac{m}{2^{k}} \leq 1 \), also wenn \( k \geq \log_2 m \).

	      Die Summe wird dann:
	      \[
		      \sum_{i=0}^{k-1} \left( \log_2 m - i \right ) = k \log_2 m - \frac{k(k-1)}{2}
	      \]

	      Da \( k = \lceil \log_2 m \rceil \), ist \( k = \mathcal{O}(\log m) \).

	      Also ist:
	      \[
		      S(m) = S(1) + \mathcal{O}\left( (\log m) \log m - (\log m)^{2} \right ) = \mathcal{O}\left( (\log m)^{2} \right )
	      \]

	      Da \( S(1) \) eine Konstante ist, folgt:
	      \[
		      T(n) = S(m) = \mathcal{O}\left( (\log \log n)^{2} \right )
	      \]

\end{enumerate}

% ========== End answering questions here

\end{document}
