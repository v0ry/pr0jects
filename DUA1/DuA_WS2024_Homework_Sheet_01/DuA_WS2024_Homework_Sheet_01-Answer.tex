\documentclass[11pt]{article}

\newcommand{\numpy}{{\tt numpy}}    % tt font for numpy

\topmargin -.5in
\textheight 9in
\oddsidemargin -.25in
\evensidemargin -.25in
\textwidth 7in

\usepackage{amsmath}
\usepackage{physics}
\begin{document}

% ========== Edit your name here
\author{Konstantin Krasser}
\title{Datenstrukturen und Algorithmen, WS2024, Übungsblatt}
\maketitle

\medskip

% ========== Begin answering questions here
\begin{enumerate}

	\item
	      \textbf{Aufgabe 1 (2 Points).} Prove that \(\sum_{i=1}^n i \in \mathcal{O}\left(n^2\right)\).

	      \textbf{Lösung:}

	      The sum \(\sum_{i=1}^n i\) is known to be equal to \(\frac{n(n+1)}{2}\). Simplifying, we have:
	      \[
		      \sum_{i=1}^n i = \frac{n(n+1)}{2} \approx \frac{n^2}{2}
	      \]
	      Since \(\frac{n^2}{2} \leq C \cdot n^2\) for some constant \(C > 0\) and large \(n\), it follows that:
	      \[
		      \sum_{i=1}^n i \in \mathcal{O}(n^2)
	      \]

	\item
	      \textbf{Aufgabe 2 (2 Points).} Prove or disprove that \(2^{2n} \in \mathcal{O}\left(2^n\right)\).

	      \textbf{Lösung:}

	      We consider the limit:
	      \[
		      \lim_{n \to \infty} \frac{2^{2n}}{2^n} = \lim_{n \to \infty} 2^n = \infty
	      \]
	      Since the limit goes to infinity, \(2^{2n}\) grows faster than \(2^n\). Therefore, \(2^{2n} \not\in \mathcal{O}(2^n)\).

	\item
	      \textbf{Aufgabe 3 (2 Points).} Prove or disprove using the limit criterion: \(\sqrt{n} = \mathcal{O}(\log n)\).

	      \textbf{Lösung:}

	      We evaluate the limit:
	      \[
		      \lim_{n \to \infty} \frac{\sqrt{n}}{\log n}
	      \]
	      As \(n \to \infty\), \(\sqrt{n}\) grows much faster than \(\log n\). Thus:
	      \[
		      \lim_{n \to \infty} \frac{\sqrt{n}}{\log n} = \infty
	      \]
	      Since the limit is infinite, \(\sqrt{n} \not\in \mathcal{O}(\log n)\).

	\item
	      \textbf{Aufgabe 4 (2 Points).} Let \(f, f^{\prime}, g, g^{\prime}: \mathbb{N} \rightarrow \mathbb{R}^{+}\) such that \(f \in \mathcal{O}(g)\) and \(f^{\prime} \in \mathcal{O}\left(g^{\prime}\right)\). Show that:

	      \[
		      f f^{\prime} \in \mathcal{O}\left(g g^{\prime}\right)
	      \]

	      Does this statement also hold analogously for asymptotically tight bounds?

	      \textbf{Lösung:}

	      By definition, there exist constants \(C_1, C_2 > 0\) and \(n_0, m_0 \in \mathbb{N}\) such that:
	      \[
		      f(n) \leq C_1 g(n) \quad \text{and} \quad f^{\prime}(n) \leq C_2 g^{\prime}(n) \quad \text{for all } n \geq \max(n_0, m_0)
	      \]
	      Multiplying these inequalities:
	      \[
		      f(n) \cdot f^{\prime}(n) \leq C_1 \cdot C_2 \cdot g(n) \cdot g^{\prime}(n)
	      \]
	      Thus, \(f f^{\prime} \in \mathcal{O}(g g^{\prime})\).

	      For asymptotically tight bounds (\(\Theta\)), if \(f \in \Theta(g)\) and \(f' \in \Theta(g')\), then:
	      \[
		      f f' \in \Theta(g g')
	      \]
	      This holds analogously.

	\item
	      \textbf{Aufgabe 5 (2 Points).} Prove or disprove that \(1+\sum_{k=2}^{\frac{n}{2}} \log(2k) \in \mathcal{O}(n \log n)\).

	      \textbf{Lösung:}

	      We analyze the sum:
	      \[
		      \sum_{k=2}^{\frac{n}{2}} \log(2k)
	      \]
	      This can be approximated by \(\sum_{k=2}^{\frac{n}{2}} \log n\), as \(\log(2k) \leq \log n\). Therefore:
	      \[
		      \sum_{k=2}^{\frac{n}{2}} \log(2k) \leq \frac{n}{2} \log n = \mathcal{O}(n \log n)
	      \]
	      Thus, \(1+\sum_{k=2}^{\frac{n}{2}} \log(2k) \in \mathcal{O}(n \log n)\).

	\item
	      \textbf{Aufgabe 6 (2 Points).} Show that \(\sum_{i=0}^{\log_2(n)-1} 8^i \in O\left(n^3\right)\).

	      \textbf{Lösung:}

	      The sum is a geometric series:
	      \[
		      \sum_{i=0}^{\log_2(n)-1} 8^i = \frac{8^{\log_2(n)} - 1}{8 - 1} \approx \frac{8^{\log_2(n)}}{7}
	      \]
	      Since \(8^{\log_2(n)} = n^{\log_2(8)} = n^3\), we have:
	      \[
		      \sum_{i=0}^{\log_2(n)-1} 8^i \in O(n^3)
	      \]

\end{enumerate}


\end{document}

